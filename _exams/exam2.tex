% Exam Template for UMTYMP and Math Department courses
%
% Using Philip Hirschhorn's exam.cls: http://www-math.mit.edu/~psh/#ExamCls
%
% run pdflatex on a finished exam at least three times to do the grading table on front page.
%
%%%%%%%%%%%%%%%%%%%%%%%%%%%%%%%%%%%%%%%%%%%%%%%%%%%%%%%%%%%%%%%%%%%%%%%%%%%%%%%%%%%%%%%%%%%%%%

% These lines can probably stay unchanged, although you can remove the last
% two packages if you're not making pictures with tikz.
\documentclass[11pt]{exam}
\RequirePackage{amssymb, amsfonts, amsmath, latexsym, verbatim, xspace, setspace}
\RequirePackage{tikz, pgflibraryplotmarks}

% By default LaTeX uses large margins.  This doesn't work well on exams; problems
% end up in the "middle" of the page, reducing the amount of space for students
% to work on them.
\usepackage[margin=1in]{geometry}
\usepackage{enumerate}
\usepackage{amsthm}

\theoremstyle{definition}
\newtheorem{soln}{Solution}

% Here's where you edit the Class, Exam, Date, etc.
\newcommand{\class}{Math 414 Section 1}
\newcommand{\term}{Fall 2022}
\newcommand{\examnum}{Exam II}
\newcommand{\examdate}{October 26, 2021}
\newcommand{\timelimit}{110 Minutes}
\newcommand{\sinc}{\text{sinc}}
\newcommand{\ol}[1]{\overline{#1}}

% For an exam, single spacing is most appropriate
\singlespacing
% \onehalfspacing
% \doublespacing

% For an exam, we generally want to turn off paragraph indentation
\parindent 0ex

\begin{document} 

% These commands set up the running header on the top of the exam pages
\pagestyle{head}
\firstpageheader{}{}{}
\runningheader{\class}{\examnum\ - Page \thepage\ of \numpages}{\examdate}
\runningheadrule

\begin{flushright}
\begin{tabular}{p{2.8in} r l}
\textbf{\class} & \textbf{Name (Print):} & \makebox[2in]{\hrulefill}\\
\textbf{\term} &&\\
\textbf{\examnum} & \textbf{Student ID:}&\makebox[2in]{\hrulefill}\\
\textbf{\examdate} &&\\
\textbf{Time Limit: \timelimit} % & Teaching Assistant & \makebox[2in]{\hrulefill}
\end{tabular}\\
\end{flushright}
\rule[1ex]{\textwidth}{.1pt}


This exam contains \numpages\ pages (including this cover page) and
\numquestions\ problems.  Check to see if any pages are missing.  Enter
all requested information on the top of this page, and put your initials
on the top of every page, in case the pages become separated.\\

You may \textit{not} use your books or notes on this exam.

You are required to show your work on each problem on this exam.  The following rules apply:\\

\begin{minipage}[t]{3.7in}
\vspace{0pt}
\begin{itemize}

%\item \textbf{If you use a ``fundamental theorem'' you must indicate this} and explain
%why the theorem may be applied.

\item \textbf{Organize your work}, in a reasonably neat and coherent way, in
the space provided. Work scattered all over the page without a clear ordering will 
receive very little credit.  

\item \textbf{Mysterious or unsupported answers will not receive full
credit}.  A correct answer, unsupported by calculations, explanation,
or algebraic work will receive no credit; an incorrect answer supported
by substantially correct calculations and explanations might still receive
partial credit.  This especially applies to limit calculations.

\item If you need more space, use the back of the pages; clearly indicate when you have done this.

\item \textbf{Box Your Answer} where appropriate, in order to clearly indicate what you consider the answer to the question to be.
\end{itemize}

Do not write in the table to the right.
\end{minipage}
\hfill
\begin{minipage}[t]{2.3in}
\vspace{0pt}
%\cellwidth{3em}
\gradetablestretch{2}
\vqword{Problem}
\addpoints % required here by exam.cls, even though questions haven't started yet.	
\gradetable[v]%[pages]  % Use [pages] to have grading table by page instead of question

\end{minipage}
\newpage % End of cover page

%%%%%%%%%%%%%%%%%%%%%%%%%%%%%%%%%%%%%%%%%%%%%%%%%%%%%%%%%%%%%%%%%%%%%%%%%%%%%%%%%%%%%
%
% See http://www-math.mit.edu/~psh/#ExamCls for full documentation, but the questions
% below give an idea of how to write questions [with parts] and have the points
% tracked automatically on the cover page.
%
%
%%%%%%%%%%%%%%%%%%%%%%%%%%%%%%%%%%%%%%%%%%%%%%%%%%%%%%%%%%%%%%%%%%%%%%%%%%%%%%%%%%%%%

\begin{questions}

\addpoints

\question[10]\mbox{}
\textbf{TRUE or FALSE!}  Write  TRUE if the statement is true.  Otherwise, write FALSE.  Your response should be in ALL CAPS.  No justification is required.
\begin{enumerate}[(a)]
\item  Given any subset $A$ of $\mathbb{R}$, $A$ is open in its relative topology.
\vspace{1.2in}
\item  The product topology on $\mathbb R\times\mathbb R$ and the Euclidean topology on $\mathbb R\times \mathbb R$ are the same.
\vspace{1.2in}
\item  The Cantor set is compact.
\vspace{1.2in}
\item  Let $P$ be a poset.  If $m\in P$ is a maximal element, then $m \geq x$ for all $x\in P$.
\vspace{1.2in}
\item  Let $X$ be compact.  Then for any continuous, real-valued function $f$ on $X$, there exists $a\in X$ such that $f(a) \geq f(x)$ for all $x\in X$.
\end{enumerate}


\newpage
\question[10]\mbox{} % definitions problem
\begin{enumerate}[(a)]
\item  Let $X$ and $Y$ be topological spaces.  Write down the definition of the topology on $X\times Y$.
\vspace{2in}
\item  Write down the Heine-Borel Theorem.
\vspace{2in}
\item  Write down Zorn's Lemma.
\vspace{2in}
\item  Write down the definition of a compact topological space.
\end{enumerate}

\newpage
\question[10]\mbox{} 
\begin{enumerate}[(a)]
\item  Give an example of a topological space $X$ with a point $x\in X$ whose singleton set $\{x\}$ is not closed.
\vspace{1.5in}
\item  Give an example of a topological space which is Lindelof but not second countable.
\vspace{1.5in}
\item  Suppose $$X = \{1,2,3\}.$$  Give three different examples of topologies on $X$.
\end{enumerate}

\newpage
\question[10]\mbox{}
You are walking in the forest at 2AM.  While passing a magical pond, you spot a frog and a toad having an intense debate about which subspaces of $\mathbb R$ are homeomorphic.
Given your history with these two mystical amphibians, you decide to help.
For each of the following pairs of subspaces of $\mathbb R$, determine \textbf{with proof} whether the spaces are homeomorphic.

\begin{enumerate}[(a)]
\item closed interval $[0,1]$ and $\mathbb Q$
\vspace{3.8in}
\item $\mathbb Q$ and $\mathbb Z$
\end{enumerate}

\newpage
\question[10]

In class, we proved the following theorem.

\textbf{Theorem:}  Let $X$ be a compact topological space and suppose $A\subseteq X$ is closed.  Then $A$ is compact.

Reprove this theorem here.

\newpage
\question[10]

Let $X$ be a set with the cofinite topology, where the open sets are $\varnothing$ and complements of finite sets.

Prove that $X$ is compact.

\newpage
\question[10]

Consider two separate topologies on $\mathbb{Z}\times \mathbb{Z}$
\begin{itemize}
\item the cofinite topology;
\item the product topology from the product of the cofinite topology on $\mathbb{Z}$ with the cofinite topology on $\mathbb{Z}$.
\end{itemize}

Are these topologies the same?  Is one topology stronger (finer) than the other?  Justify your answers.


\end{questions}

\end{document}
