% Exam Template for UMTYMP and Math Department courses
%
% Using Philip Hirschhorn's exam.cls: http://www-math.mit.edu/~psh/#ExamCls
%
% run pdflatex on a finished exam at least three times to do the grading table on front page.
%
%%%%%%%%%%%%%%%%%%%%%%%%%%%%%%%%%%%%%%%%%%%%%%%%%%%%%%%%%%%%%%%%%%%%%%%%%%%%%%%%%%%%%%%%%%%%%%

% These lines can probably stay unchanged, although you can remove the last
% two packages if you're not making pictures with tikz.
\documentclass[11pt]{exam}
\RequirePackage{amssymb, amsfonts, amsmath, latexsym, verbatim, xspace, setspace}
\RequirePackage{tikz, pgflibraryplotmarks}

% By default LaTeX uses large margins.  This doesn't work well on exams; problems
% end up in the "middle" of the page, reducing the amount of space for students
% to work on them.
\usepackage[margin=1in]{geometry}
\usepackage{enumerate}
\usepackage{amsthm}

\theoremstyle{definition}
\newtheorem{soln}{Solution}

% Here's where you edit the Class, Exam, Date, etc.
\newcommand{\class}{Math 414 Section 1}
\newcommand{\term}{Fall 2022}
\newcommand{\examnum}{Exam III}
\newcommand{\examdate}{October 26, 2021}
\newcommand{\timelimit}{50 Minutes}
\newcommand{\sinc}{\text{sinc}}
\newcommand{\ol}[1]{\overline{#1}}

% For an exam, single spacing is most appropriate
\singlespacing
% \onehalfspacing
% \doublespacing

% For an exam, we generally want to turn off paragraph indentation
\parindent 0ex

\begin{document} 

% These commands set up the running header on the top of the exam pages
\pagestyle{head}
\firstpageheader{}{}{}
\runningheader{\class}{\examnum\ - Page \thepage\ of \numpages}{\examdate}
\runningheadrule

\begin{flushright}
\begin{tabular}{p{2.8in} r l}
\textbf{\class} & \textbf{Name (Print):} & \makebox[2in]{\hrulefill}\\
\textbf{\term} &&\\
\textbf{\examnum} & \textbf{Student ID:}&\makebox[2in]{\hrulefill}\\
\textbf{\examdate} &&\\
\textbf{Time Limit: \timelimit} % & Teaching Assistant & \makebox[2in]{\hrulefill}
\end{tabular}\\
\end{flushright}
\rule[1ex]{\textwidth}{.1pt}


This exam contains \numpages\ pages (including this cover page) and
\numquestions\ problems.  Check to see if any pages are missing.  Enter
all requested information on the top of this page, and put your initials
on the top of every page, in case the pages become separated.\\

You may \textit{not} use your books or notes on this exam.

You are required to show your work on each problem on this exam.  The following rules apply:\\

\begin{minipage}[t]{3.7in}
\vspace{0pt}
\begin{itemize}

%\item \textbf{If you use a ``fundamental theorem'' you must indicate this} and explain
%why the theorem may be applied.

\item \textbf{Organize your work}, in a reasonably neat and coherent way, in
the space provided. Work scattered all over the page without a clear ordering will 
receive very little credit.  

\item \textbf{Mysterious or unsupported answers will not receive full
credit}.  A correct answer, unsupported by calculations, explanation,
or algebraic work will receive no credit; an incorrect answer supported
by substantially correct calculations and explanations might still receive
partial credit.  This especially applies to limit calculations.

\item If you need more space, use the back of the pages; clearly indicate when you have done this.

\item \textbf{Box Your Answer} where appropriate, in order to clearly indicate what you consider the answer to the question to be.
\end{itemize}

Do not write in the table to the right.
\end{minipage}
\hfill
\begin{minipage}[t]{2.3in}
\vspace{0pt}
%\cellwidth{3em}
\gradetablestretch{2}
\vqword{Problem}
\addpoints % required here by exam.cls, even though questions haven't started yet.	
\gradetable[v]%[pages]  % Use [pages] to have grading table by page instead of question

\end{minipage}
\newpage % End of cover page

%%%%%%%%%%%%%%%%%%%%%%%%%%%%%%%%%%%%%%%%%%%%%%%%%%%%%%%%%%%%%%%%%%%%%%%%%%%%%%%%%%%%%
%
% See http://www-math.mit.edu/~psh/#ExamCls for full documentation, but the questions
% below give an idea of how to write questions [with parts] and have the points
% tracked automatically on the cover page.
%
%
%%%%%%%%%%%%%%%%%%%%%%%%%%%%%%%%%%%%%%%%%%%%%%%%%%%%%%%%%%%%%%%%%%%%%%%%%%%%%%%%%%%%%

\begin{questions}

\addpoints

\question[10]\mbox{}
\textbf{TRUE or FALSE!}  Write  TRUE if the statement is true.  Otherwise, write FALSE.  Your response should be in ALL CAPS.  No justification is required.
\begin{enumerate}[(a)]
\item  
A topological space consisting of a single point must be connected.
\vspace{1.2in}
\item  
The set $\mathbb{R}$ with the cofinite topology is Hausdorff.
\vspace{1.2in}
\item  
A metric space must be $T_4$.
\vspace{1.2in}
\item  
In a topological space, every open cover has a Lebesgue number.
\vspace{1.2in}
\item  
Any set with the cofinite topology is compact.
\end{enumerate}


\newpage
\question[10]\mbox{} % definitions problem
\begin{enumerate}[(a)]
\item  Write down what it means for a topological space $X$ to be disconnected and what it means for a subset $A\subseteq X$ to be disconnected.
\vspace{2in}
\item  Write down the definition of a metric space being sequentially compact.
\vspace{2in}
\item  Write down Urysohn's Lemma.
\vspace{2in}
\item  Write down the Bolzano-Weierstrass Theorem.
\end{enumerate}

\newpage
\question[10]\mbox{} 
\begin{enumerate}[(a)]
\item  Give an example of an infinite connected subset of $\mathbb{R}$.
\vspace{1.5in}
\item  Give an example of an infinite compact subset of $\mathbb{R}$.
\vspace{1.5in}
\item  Give an example of a topological space which is $T_0$ but not $T_1$.
\vspace{1.5in}
\item  Give an example of a topological space with a compact subset which is not closed.
\vspace{1.5in}
\item  Give an example of a topological space with just two points which is connected.
\end{enumerate}

\newpage
\question[10]\mbox{}
In class we proved that if $A\subseteq X$ is connected, then the closure $\overline{A}$ is also connected.
Reprove that fact here.

\newpage
\question[10]\mbox{}
For each of the following pairs of topological spaces, explain why the two topological spaces are not homeomorphic.  Assume the Euclidean topology.

\begin{enumerate}[(a)]
\item the Cantor set and the set of all irrational numbers in $[0,1]$
\vspace{1.5in}
\item $\mathbb Z$ and the open interval $(0,1)$
\vspace{1.5in}
\item the Cantor set and the set $[0,1]$
\vspace{1.5in}
\item the closed interval $[0,1]$ and the closed unit square $[0,1]\times [0,1]$
\end{enumerate}

\newpage
\textbf{Takehome Portion!}\\
Detach this portion of the exam and take it home with you.

\textbf{Problem 1:}
Let $X$ be a metric space with metric $d$.
\begin{enumerate}[(a)]
\item Prove that the function 
$$\rho((x_1,y_1),(x_2,y_2)) = \sqrt{d(x_1,x_2)^2+d(y_1,y_2)^2}$$
is a metric on the product set $X\times X$.
\item Prove that the topology defined by $\rho$ is the same as the product topology on $X\times X$.
\end{enumerate}

\textbf{Problem 2:}
Consider the unit square minus one point
$$X = \{(x,y)\in\mathbb R^2: 0\leq x,y\leq 1,\ \ (x,y)\neq(1/2,1/2)\}.$$
Prove that $X$ is connected.

\end{questions}

\end{document}
