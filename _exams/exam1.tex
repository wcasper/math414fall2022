% Exam Template for UMTYMP and Math Department courses
%
% Using Philip Hirschhorn's exam.cls: http://www-math.mit.edu/~psh/#ExamCls
%
% run pdflatex on a finished exam at least three times to do the grading table on front page.
%
%%%%%%%%%%%%%%%%%%%%%%%%%%%%%%%%%%%%%%%%%%%%%%%%%%%%%%%%%%%%%%%%%%%%%%%%%%%%%%%%%%%%%%%%%%%%%%

% These lines can probably stay unchanged, although you can remove the last
% two packages if you're not making pictures with tikz.
\documentclass[11pt]{exam}
\RequirePackage{amssymb, amsfonts, amsmath, latexsym, verbatim, xspace, setspace}
\RequirePackage{tikz, pgflibraryplotmarks}

% By default LaTeX uses large margins.  This doesn't work well on exams; problems
% end up in the "middle" of the page, reducing the amount of space for students
% to work on them.
\usepackage[margin=1in]{geometry}
\usepackage{enumerate}
\usepackage{amsthm}

\theoremstyle{definition}
\newtheorem{soln}{Solution}

% Here's where you edit the Class, Exam, Date, etc.
\newcommand{\class}{Math 414 Section 1}
\newcommand{\term}{Fall 2022}
\newcommand{\examnum}{Exam I}
\newcommand{\examdate}{September 26, 2021}
\newcommand{\timelimit}{110 Minutes}
\newcommand{\sinc}{\text{sinc}}
\newcommand{\ol}[1]{\overline{#1}}

% For an exam, single spacing is most appropriate
\singlespacing
% \onehalfspacing
% \doublespacing

% For an exam, we generally want to turn off paragraph indentation
\parindent 0ex

\begin{document} 

% These commands set up the running header on the top of the exam pages
\pagestyle{head}
\firstpageheader{}{}{}
\runningheader{\class}{\examnum\ - Page \thepage\ of \numpages}{\examdate}
\runningheadrule

\begin{flushright}
\begin{tabular}{p{2.8in} r l}
\textbf{\class} & \textbf{Name (Print):} & \makebox[2in]{\hrulefill}\\
\textbf{\term} &&\\
\textbf{\examnum} & \textbf{Student ID:}&\makebox[2in]{\hrulefill}\\
\textbf{\examdate} &&\\
\textbf{Time Limit: \timelimit} % & Teaching Assistant & \makebox[2in]{\hrulefill}
\end{tabular}\\
\end{flushright}
\rule[1ex]{\textwidth}{.1pt}


This exam contains \numpages\ pages (including this cover page) and
\numquestions\ problems.  Check to see if any pages are missing.  Enter
all requested information on the top of this page, and put your initials
on the top of every page, in case the pages become separated.\\

You may \textit{not} use your books or notes on this exam.

You are required to show your work on each problem on this exam.  The following rules apply:\\

\begin{minipage}[t]{3.7in}
\vspace{0pt}
\begin{itemize}

%\item \textbf{If you use a ``fundamental theorem'' you must indicate this} and explain
%why the theorem may be applied.

\item \textbf{Organize your work}, in a reasonably neat and coherent way, in
the space provided. Work scattered all over the page without a clear ordering will 
receive very little credit.  

\item \textbf{Mysterious or unsupported answers will not receive full
credit}.  A correct answer, unsupported by calculations, explanation,
or algebraic work will receive no credit; an incorrect answer supported
by substantially correct calculations and explanations might still receive
partial credit.  This especially applies to limit calculations.

\item If you need more space, use the back of the pages; clearly indicate when you have done this.

\item \textbf{Box Your Answer} where appropriate, in order to clearly indicate what you consider the answer to the question to be.
\end{itemize}

Do not write in the table to the right.
\end{minipage}
\hfill
\begin{minipage}[t]{2.3in}
\vspace{0pt}
%\cellwidth{3em}
\gradetablestretch{2}
\vqword{Problem}
\addpoints % required here by exam.cls, even though questions haven't started yet.	
\gradetable[v]%[pages]  % Use [pages] to have grading table by page instead of question

\end{minipage}
\newpage % End of cover page

%%%%%%%%%%%%%%%%%%%%%%%%%%%%%%%%%%%%%%%%%%%%%%%%%%%%%%%%%%%%%%%%%%%%%%%%%%%%%%%%%%%%%
%
% See http://www-math.mit.edu/~psh/#ExamCls for full documentation, but the questions
% below give an idea of how to write questions [with parts] and have the points
% tracked automatically on the cover page.
%
%
%%%%%%%%%%%%%%%%%%%%%%%%%%%%%%%%%%%%%%%%%%%%%%%%%%%%%%%%%%%%%%%%%%%%%%%%%%%%%%%%%%%%%

\begin{questions}

\addpoints

\question[10]\mbox{}
\textbf{TRUE or FALSE!}  Write  TRUE if the statement is true.  Otherwise, write FALSE.  Your response should be in ALL CAPS.  No justification is required.
\begin{enumerate}[(a)]
\item  If a set $A$ is uncountable, then $A$ has the same cardinality as $\mathbb R$.
\vspace{1.2in}
\item  The union of infinitely many open sets is still open.
\vspace{1.2in}
\item  If $f: X\rightarrow Y$ is a function between metric spaces, then the preimage of any closed set must be closed.
\vspace{1.2in}
\item  The $3$-adic distance between $5$ and $14$ is $1/9$.
\vspace{1.2in}
\item  If $A\subseteq X$ is subset of the metric space $X$, then either $A$ or $A'$ must be closed.
\vspace{1.2in}
\end{enumerate}

\newpage
\question[10]\mbox{} 
For each of the following, either prove that the function is a metric on $\mathbb R^2$ or explain why it is not.
\begin{enumerate}[(a)]
\item  $d((x_1,y_1),(x_2,y_2)) = |x_1-x_2|\cdot |y_1-y_2|$
\vspace{1.5in}
\item  $d((x_1,y_1),(x_2,y_2)) = |x_1-y_1| + |x_2-y_2|$
\vspace{1.5in}
\item  $d((x_1,y_1),(x_2,y_2)) = \frac{1}{1 + |x_1-x_2| + |y_1-y_2|}$
\vspace{1.5in}
\item  $d((x_1,y_1),(x_2,y_2)) = \text{max}\{|x_1-x_2|,|y_1-y_2|\}$
\end{enumerate}

\newpage
\question[10]\mbox{} % definitions problem
\begin{enumerate}[(a)]
\item  Let $X$ and $Y$ be metric spaces.  Write the definition of a function $f: X\rightarrow Y$ being uniformly continuous.
\vspace{2in}
\item  Give an example of a metric space $X$ and a sequence of continuous, real-valued functions on $X$ which converge pointwise to a discontinuous function on $X$.  Make sure to indicate the metric on $X$.
\vspace{2.0in}
\item  State the Cauchy-Schwarz inequality for $\mathbb R^n$ and describe how we used it in class.
\end{enumerate}

\newpage
\question[10]\mbox{}
\begin{enumerate}[(a)]
\item  Write the definition of two sets having the same cardinality.
\vspace{1in}
\item  State the Cantor-Schroeder-Bernstein theorem.
\vspace{1in}
\item  Show that the set
$$\mathcal F = \{f: \{0,1\}\rightarrow\mathbb N\}.$$
of functions from $\{0,1\}$ to $\mathbb N$ is countable.
\end{enumerate}

\newpage
\question[10]
Let $X$ be a metric space.
\begin{enumerate}[(a)]
\item Write the definition of a limit point of a subset $A\subseteq X$.
\vspace{1in}
\item Write the definition of a boundary point of a subset $A\subseteq X$.
\vspace{1in}
\item Prove that the closed ball $\overline{B_r}(x)$ is a closed set.
\end{enumerate}

\newpage
\question[10]
Let $X$ be a set and suppose that $d$ and $\rho$ are two metrics on $X$ satisfying the property that there exists a constant $M>0$ with

$$\frac{1}{M}\rho(x,y)\leq d(x,y)\leq M\rho(x,y)\quad\text{for all }x,y\in X.$$

Prove that $U\subseteq X$ is open in the metric space $(X,\rho)$ if and only if it is open in $(X,d)$.

\newpage
\question[10]
Let $X = \mathbb R$ with the discrete metric and let $C\subseteq X$ be the Cantor set.
\begin{enumerate}[(a)]
\item  Determine the interior $\text{int}(C)$
\vspace{3in}
\item  Determine the boundary $\partial C$
\end{enumerate}


\end{questions}
\end{document}
